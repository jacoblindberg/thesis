
\input{summarystats.tex}

\newpage

\section{Model comparison}\label{model-comparison}

\emph{In this section we compare the two models. Firstly, we compare the fit (in-sample) and then their forcasts (out-of-sample).}

\subsection{Fit Comparison}\label{fit-comparison}

We use the word \emph{fit} when discussing in-sample properties and the word \emph{forecast} when discussing out-of-sample forecasts. This fit comparison is not a major part of the thesis but sould be seen as a first step. We fit two models to the estimation window\footnote{Questino to tutor. Should I display the fit for the entire dataset or only using the estimation window? I think it would be more fair to do the latter. Then we dont ``peek'' at the data before doing forecasting.} conclude they are different and then move on to compare their out-of-sample forecasts. Therefore this subsection should be seen partly as a way of explaining the differences between the two models, and partly as an introduction to the forecast comparison because that is what we are most interested in.

Output  from the \emph{fits} are on page \pageref{sec-fit.acd-fit.garch.r} in Appendix. In the paragraphs below we will comment this output.
